%% stripped down version of "bare_jrnl.tex" for use in EE3370. 
%% Original version has very good comments for use. You should check it out at
%% http://www.ieee.org/publications_standards/publications/authors/authors_journals.html
%% I run GNU/Linux so I downloaded the "Unix LaTeX2e Transactions Style File"
%% package and based my work off of the sample tex file named "bare_jrnl.tex".

% original author info below (this guy's a rockstar for making his comments 
% so easy to use :P)
%% 2007/01/11
%% by Michael Shell
%% see http://www.michaelshell.org/
%% for current contact information.

\documentclass[journal]{IEEEtran}
% make sure "IEEEtran.cls" is in the path of the tex file you are working on

%graphics package for adding images
\ifCLASSINFOpdf
  \usepackage[pdftex]{graphicx}
\else
   \usepackage[dvips]{graphicx}
\fi

%math package for math equations
\usepackage[cmex10]{amsmath}

%float package for putting images where I fucking tell them to go
\usepackage{float}

%for sourcecode
%\usepackage{listings}
%\lstset{breaklines=true,language=gnuplot,basicstyle=\scriptsize,showspaces=false,showstringspaces=false}

%For hyperlinks
\usepackage{hyperref}

%For strikethrough ([normalem] option preserves italicising via emph)
\usepackage[normalem]{ulem}

\begin{document}

% paper title
% can use linebreaks \\ within to get better formatting as desired
\title{Research Journal for Preston Maness during Senior Design I: 
A Hardware Random Number Generator}
\author{Preston~Maness}

% header
\markboth{Texas State University, Dr. Stapleton, EE4390, Senior Design I}%
{}

% make the title area
\maketitle

% Give the abstract of your lab here
\begin{abstract}
Just a dummy abstract here. Probably won't actually use an abstract for the 
journal.
\end{abstract}

\tableofcontents

\section{Sunday September 15 2013}

Was able to get dieharder compiled and running. Can feed in both /dev/urandom 
and a binary file of random, unsigned, 32-bit int's generated from Perl's 
rand() function. While /dev/urandom is passing most tests --it is marked
WEAK in others-- it seems that generating even a million random ints is 
not sufficient for dieharder. It will generate output stating that it "rewound"
the file anywhere from tens of times to hundreds of times. Of course when 
it rewinds the file it is no longer "random" and so the tests fail. Good to 
know.

As well, it is becoming apparent that I will need to develop a rigourous 
statistical understanding of randomness. I should also familiarize myself with 
the GNU Scientific Library, as dieharder integrates tightly with it.

Regardless, I have the RNG stress-tester up and running. Now I need to focus 
on perhaps making a randomness bitstream a la /dev/customRandom, as our 
RNG will ultimately be speaking over a serial line and into the host that 
will then put the raw bits here.

\section{Thursday September 19 2013}

Found an outstanding link on random noise generation using avalanche 
breakdown. He even utilized ngspice in the simulation process! This should 
prove to be an outstanding guidepost. Judging by his work, there's a good 
chance that avalanche noise might be the best source to work with if 
construction by hand is a requirement.

\url{http://holdenc.altervista.org/avalanche/}

Still reading. It looks like he's covered all the fundamental bases I wanted 
to cover. If his work proves to speed up mine considerably, I should 
consider investigating parallelizing these designs and having a 
microcontroller for de-skewing and de-biasing this semester, rather than 
second semester.

Going to test his netlist on my installation of ngspice now. Getting similar 
results. That's good.

However, looking at his schematic it looks like he's throwing enough voltage 
at the 2N3904's to kill them over time, though I could be reading it wrong. I 
suspect you'd need some heat sinking to keep them cool enough, but he's not 
exactly throwing a lot of current at them either. 

When the current is changed, Vazzana states that ``the noise `pattern' seems to
change on the oscilloscope.'' Will need to investigate this fully.

\url{http://http.developer.nvidia.com/GPUGems3/gpugems3_ch37.html}

The NVIDIA link has a nice list of references I should look into.

\section{September 23 2013}

Been thinking about how to get this rapidly prototyped. The end of September 
is a lot closer than it seems. I don't want to fall behind schedule. For 
now, I'm looking into having an externally powered RNG that outputs its serial 
bitstream to something like a Rasberry Pi or an Arduino UNO. I own both of 
these pieces of hardware. I'm curious to see if I can accomplish debiasing 
and decorrelating with either of these bits of kit. I can then push out the 
purified bitstream over USB. Both of these devices have libraries available to 
take an input bitstream and send out a modified stream to a host via USB.

\url{https://github.com/infomaniac50/Random}

It looks like infomaniac50 is several steps ahead of me. Sidenote: It seems 
everyone is several steps ahead of me. I should think about how I can 
distinguish my work. The availability of the work of others means I should be 
able to get up and prototyping within the week. How can I take advantage of 
the additional time to produce a unique addition to this body of work? I'm 
leaning toward using several RNGs in parallel and then using a suitably fast 
uC to manipulate it into a single, high-speed bitstream over USB 2.0.

Anyway... infomaniac50's got an arduino library that takes the input from an
RNG and debiases it. It should be simple enough to extend this library and 
have it output the bitstream over USB to a host. As well, it's licensed 
under the CC-AT-SA 3.0 Unported license. I don't know if this is compatible 
off the top of my head; my understanding is that CC licenses aren't really 
meant to cover code.

After that, it's just a matter of familiarizing myself with the internals of 
the Linux kernel to get a bitstream from a /dev/USB device into /dev/urandom 
or some other custom entropy pool (like /dev/myCustomEntropyPool).

Having some pretty graphs that show how much entropy is in this pool at any 
given time would be nice. Have a graph that shows how much entropy is used 
up during entropy-intensive processes. Ideally, we shouldn't be losing much 
entropy --that is to say, the hardware RNG should be able to keep up. In 
reality, I suspect my first Marks to not be able to keep up with dieharder's 
entropy demands.

\subsection{Later in the day inbetween classes...}

The title for this paper seems appropriate enough:

\url{http://dasgupab.faculty.udmercy.edu/Dasgupta-JSfinal.pdf}

``Mathematical Foundations of Randomness.'' Good source.

\section{Monday November 4 2013}

Who am I kidding. October flew by with hardly any work getting done. The
other four courses have been eating up all my time. However, I'm doing well
in them and plan on shifting gears for a little bit. I can afford to let
a few things slide. But really, I need to get a physical prototype up and
running ASAP. I've been running through simulations, and thought of a
rather interesting chicken and egg issue:

If I'm building a hardware RNG, then how can I trust any simulation designed
to test such an RNG without having a hardware RNG providing the noise to
the simulator? How does the simulator handle noise at all? Going to add
this to the research log. I think this is a good example of how simulation
only goes so far.

Ok. Now it's officially Monday. I've got the analog noise floor up and 
amplified to 500 mV, with what looks at least in passing to be random 
oscillation. Great! See the ``analog\_only.sch'' using gschem or view the 
net at ``analog\_only.net''. It should run fine in ngspice. 

This serves as my initial verification of the work done by Vazzana.

Useful links follow:

\begin{itemize}
\item
http://www-mdp.eng.cam.ac.uk/web/CD/engapps/geda/geda-doc/spice-sdb/netlist.html
\item
http://www.brorson.com/gEDA/SPICE/x496.html
\item
http://web.jfet.org/hw-rng.html
\end{itemize}

\section{Tuesday March 04 2014}

It's been quite a while since I updated this research log. Much of my work 
has been logged in the ``weekly-reports'' section of the repository. In any 
case, I'm running into problems with what I'm assuming is a biased bitstream.

The following paper appears VERY relevant to my interests --that is, it 
addresses many methods of debiasing/decorrelating bitstreams-- and the 
link is below:

\url{http://www1.spms.ntu.edu.sg/~kkhoongm/Entropy.pdf}

\section{Initial Research - What I've Found}

\IEEEPARstart{B}{lah} blah blah... List of sources below. Going to keep 
adding text in here so that the itemize list doesn't get shoved up into 
the fancy 'B' character of "Blah."

\begin{itemize}
\item
\cite{000321520700011n.d.} Evaluating a TRNG in hardware.
\item
\cite{000322026900007n.d.} Noise resistant TRNG. Stochastic model for parameter choicing.
\item
\cite{6082794820110401} Importance of RNG choice in GIS applications.
\item
\cite{6102549620110515} Ring-based RNG. Huh? What's that?
\item
\cite{8288132620110601} Investigating LFSR, LCG, and Blum Blum Shub on 
Xilinx FPGA
\item
\cite{8704557320120101} Non-Uniform RNG, Statistics of.
\item
\cite{8762674520130201} "Data-oriented" RNG? Not sure what this is about, but it mentions 
making distributions of random numbers with different characteristics (uniform, chi-squared, 
etc)
\item
\cite{000315974100018n.d.} GPU Accelerated Scalable Parallel RNG.
\end{itemize}

\bibliography{IEEEabrv,works-cited}{}
\bibliographystyle{IEEEtran}

\end{document}
