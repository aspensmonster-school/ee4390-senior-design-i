%% stripped down version of "bare_jrnl.tex" for use in EE3370. 
%% Original version has very good comments for use. You should check it out at
%% http://www.ieee.org/publications_standards/publications/authors/authors_journals.html
%% I run GNU/Linux so I downloaded the "Unix LaTeX2e Transactions Style File"
%% package and based my work off of the sample tex file named "bare_jrnl.tex".

% original author info below (this guy's a rockstar for making his comments 
% so easy to use :P)
%% 2007/01/11
%% by Michael Shell
%% see http://www.michaelshell.org/
%% for current contact information.

\documentclass[journal]{IEEEtran}
% make sure "IEEEtran.cls" is in the path of the tex file you are working on

%graphics package for adding images
\ifCLASSINFOpdf
  \usepackage[pdftex]{graphicx}
\else
   \usepackage[dvips]{graphicx}
\fi

%math package for math equations
\usepackage[cmex10]{amsmath}

%float package for putting images where I fucking tell them to go
\usepackage{float}

%for sourcecode
%\usepackage{listings}
%\lstset{breaklines=true,language=gnuplot,basicstyle=\scriptsize,showspaces=false,showstringspaces=false}

%For hyperlinks
\usepackage{hyperref}

%For strikethrough ([normalem] option preserves italicising via emph)
\usepackage[normalem]{ulem}

\begin{document}

% paper title
% can use linebreaks \\ within to get better formatting as desired
\title{Research Journal for Preston Maness during Senior Design I: 
A Hardware Random Number Generator}
\author{Preston~Maness}

% header
\markboth{Texas State University, Dr. Stapleton, EE4390, Senior Design I}%
{}

% make the title area
\maketitle

% Give the abstract of your lab here
\begin{abstract}
Just a dummy abstract here. Probably won't actually use an abstract for the 
journal.
\end{abstract}

\tableofcontents

\section{Sunday September 15 2013}

Was able to get dieharder compiled and running. Can feed in both /dev/urandom 
and a binary file of random, unsigned, 32-bit int's generated from Perl's 
rand() function. While /dev/urandom is passing most tests --it is marked
WEAK in others-- it seems that generating even a million random ints is 
not sufficient for dieharder. It will generate output stating that it "rewound"
the file anywhere from tens of times to hundreds of times. Of course when 
it rewinds the file it is no longer "random" and so the tests fail. Good to 
know.

As well, it is becoming apparent that I will need to develop a rigourous 
statistical understanding of randomness. I should also familiarize myself with 
the GNU Scientific Library, as dieharder integrates tightly with it.

Regardless, I have the RNG stress-tester up and running. Now I need to focus 
on perhaps making a randomness bitstream a la /dev/customRandom, as our 
RNG will ultimately be speaking over a serial line and into the host that 
will then put the raw bits here.

\section{Initial Research - What I've Found}

\IEEEPARstart{B}{lah} blah blah... List of sources below. Going to keep 
adding text in here so that the itemize list doesn't get shoved up into 
the fancy 'B' character of "Blah."

\begin{itemize}
\item
\cite{000321520700011n.d.} Evaluating a TRNG in hardware.
\item
\cite{000322026900007n.d.} Noise resistant TRNG. Stochastic model for parameter choicing.
\item
\cite{6082794820110401} Importance of RNG choice in GIS applications.
\item
\cite{6102549620110515} Ring-based RNG. Huh? What's that?
\item
\cite{8288132620110601} Investigating LFSR, LCG, and Blum Blum Shub on 
Xilinx FPGA
\item
\cite{8704557320120101} Non-Uniform RNG, Statistics of.
\item
\cite{8762674520130201} "Data-oriented" RNG? Not sure what this is about, but it mentions 
making distributions of random numbers with different characteristics (uniform, chi-squared, 
etc)
\item
\cite{000315974100018n.d.} GPU Accelerated Scalable Parallel RNG.
\end{itemize}

\bibliography{IEEEabrv,works-cited}{}
\bibliographystyle{IEEEtran}

\end{document}
